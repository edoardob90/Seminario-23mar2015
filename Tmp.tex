  Simulazioni NVT a \num{500} e \SI{700}{K}
  \vspace{0.2cm}
  \begin{columns}
   \begin{column}{.5\textwidth}
   \centering
    \includegraphics[width=\textwidth]{bulk_Lcr_500}
   \end{column}
   \begin{column}{0.5\textwidth}
   \centering
    \includegraphics[width=\textwidth]{bulk_Lcr_700}
   \end{column}
  \end{columns}
  
  
  
  \begin{frame}{Cristallizzazione}
 \only<1-2>{%
  \begin{itemize}
   \item<1-> Simulazioni $NVT$ per $\sim \SI{e2}{ps}$
   \item<1-> Temperature nell'intervallo \num{500}--\SI{850}{K}
   \item<2-> Atomo cristallino secondo il parametro d'ordine $\bar{Q}_4$ \\
      %\begin{flushright}
       %{\cit{[Sosso et al., J. Phys. Chem. C (2015)]}}
      %\end{flushright}
    \item<2-> Spessore dello \emph{slab} cristallino in funzione di $t$
      \begin{displaymath}
       L\ped{cr}=N\ped{cr}\frac{d\ped{hkl}}{2N\ped{surf}}
      \end{displaymath}
    \item<2-> Velocità di crescita $u=dL\ped{cr}/dt$
  \end{itemize}
}
\only<3>{%
  \begin{columns}
   \begin{column}{0.5\textwidth}
    \includegraphics[width=\textwidth]{Lcr_t_500}
   \end{column}
   \begin{column}{0.5\textwidth}
    \includegraphics[width=\textwidth]{Lcr_t_700}
   \end{column}
  \end{columns}
}
\end{frame}






%%% Equazioni CNT


\only<-3>{%
 \begin{columns}[c]
 \onslide<2-3>{%
  \begin{column}{0.5\textwidth}
  \centering
  \begin{displaymath}
   u\ped{kin}= \frac{6 D}{\lambda}
  \end{displaymath}
  {\ev fattore cinetico} che dipende dal coefficiente $D$
  \end{column}} % end first onslide
%
 \onslide<3>{%
  \begin{column}{0.5\textwidth}
  \centering
  \begin{displaymath}
   \Delta\mu = \Delta H_m \frac{(T_m-T)\,2T}{(T_m+T)\,T_m}
  \end{displaymath}
  {\ev differenza di potenziale chimico} tra cristallo e amorfo (formula di Thompson--Spaepen)
  \end{column}} % end second onslide
 \end{columns}
} % end only
%
 \only<4->{%
  \begin{exampleblock}{}
   \centering
   $D$ e $\Delta\mu$ hanno andamenti opposti in funzione di $T$ \\
   La funzione $u(T)$ ha perciò un massimo
  \end{exampleblock}
}