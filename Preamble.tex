%****************************************************************************************************
% PACCHETTI
%****************************************************************************************************
% Tema principale
%\usetheme{Mainz}

%\usepackage{fontspec}
%\usepackage[libertine={Ligatures=TeX,RawFeature=+onum},biolinum={Ligatures=TeX,RawFeature=+onum}]{libertineotf}

\usepackage[biolinum,sfdefault]{libertine}
\usepackage{eulervm}

\usepackage[italian]{babel}

\usepackage[T1]{fontenc}
\usepackage[utf8]{inputenc}

%\usepackage{beamerfoils}
%\usepackage{array}
%\usepackage{mathrsfs}
\usepackage{booktabs}
\usepackage{color}
\usepackage{colortbl}    
\usepackage[version=3]{mhchem} 
%\usepackage{gensymb}
\usepackage{tabularx}
%\usepackage{cancel}
\usepackage[labelformat=empty,labelsep=none,skip=1pt]{caption}
\usepackage[per-mode=symbol]{siunitx}
\usepackage{relsize,xspace}
\usepackage{bm}
\usepackage{pgfpages}

\usepackage{tikz}
\usetikzlibrary{shapes,arrows,shadows,mindmap,trees,calc}

\usepackage{appendixnumberbeamer}

\usepackage{lipsum}

%****************************************************************************************************

%\setbeamercovered{dynamic}

\def\arrowd{
  (10.75:1.1) -- (6.5:1) arc (6.25:120:1) [rounded corners=0.5] --
  (120:0.9) [rounded corners=1] -- (130:1.1) [rounded corners=0.5] --
  (120:1.3) [sharp corners] -- (120:1.2) arc (120:5.25:1.2)
  [rounded corners=1] -- (10.75:1.1) -- (6.5:1) -- cycle
}

\tikzset{
  ashadow/.style={opacity=.25, shadow xshift=0.07, shadow yshift=-0.07},
}


%****************************************************************************************************
% COLORI e STILI di TESTO
\definecolor{greendark}{RGB}{0,178,140}
\definecolor{bluegreen}{cmyk}{0.9,0.0,0.35,0.2}
\definecolor{darkblue}{rgb}{0.2,0.2,0.65}
\definecolor{themecolor}{RGB}{38,29,162} % blue


%\newcommand{\cit}{\scriptsize\color{bluegreen}}             % citazioni
\newcommand{\cit}{\scriptsize\color{themecolor!80!green}}             % citazioni
\newcommand{\tbtit}{\bf\color{themecolor!75!black}} % titoli nelle tabelle
\newcommand{\ev}{\color{themecolor}\bf}
%\renewcommand{\CancelColor}{\color{red}}
%****************************************************************************************************


%****************************************************************************************************
% TABELLE
\newcolumntype{C}[1]{>{\centering\let\newline\\\arraybackslash\hspace{0pt}}m{#1}}
\setlength{\aboverulesep}{0pt}
\setlength{\belowrulesep}{0pt}
\setlength{\extrarowheight}{.75ex}
\arrayrulecolor{themecolor!75!black}

\newcommand{\tabitem}{~~\llap{\textbullet}~~}
%****************************************************************************************************


%****************************************************************************************************
% FRECCE
\tikzset{bluearrow/.style={draw=themecolor,fill=themecolor,single arrow,drop shadow=%
                           {shadow xshift=.3ex,shadow yshift=-.3ex,color=themecolor!60!black},
                           minimum height=3.5ex,minimum width=0.1ex,single arrow head extend=0.5ex,
                           single arrow tip angle=70}}

\newcommand{\arrowup}{\tikz[baseline=-0.5ex]{\node[bluearrow,rotate=90]{};}}
\newcommand{\arrowdown}{\tikz[baseline=-0.5ex]{\node[bluearrow,rotate=-90]{};}}
\newcommand{\arrowright}{\tikz[baseline=-0.5ex]{\node[bluearrow]{};}}
\newcommand{\arrowleft}{\tikz[baseline=-0.5ex]{\node[bluearrow,rotate=180]{};}}
%****************************************************************************************************


% COMANDI PERSONALI
\newcommand{\gete}{\ce{GeTe}\xspace}

% Path per le immagini
\graphicspath{{Immagini/}{../Thesis/Immagini/Plots/}{../THESIS/Immagini/}{../THESIS/Immagini/Plots/}{../THESIS/Immagini/bulk_figures/}}

