%****************************************************************************************************
% PACCHETTI
%****************************************************************************************************
% Tema principale
\usetheme{Mainz}

\usepackage[english,italian]{babel}
\usepackage[T1]{fontenc}
\usepackage[utf8]{inputenc}
\usepackage{beamerfoils}
\usepackage{array}
\usepackage{mathrsfs}
\usepackage{tikz}  %--------------------------------------------- disegnini
\usepackage{booktabs}  %----------------------------------------- tabelle
\usepackage{color}  %-------------------------------------------- colori
\usepackage{colortbl}  %----------------------------------------- tabelle colorate    
\usepackage[version=3]{mhchem}  %-------------------------------- formule chimiche 
\usepackage{gensymb}
\usepackage{tabularx}
\usepackage{cancel}
\usepackage[labelformat=empty,labelsep=none,skip=1pt]{caption}
\usepackage[per-mode=symbol]{siunitx}
\usepackage{relsize,xspace}
\usetikzlibrary{shapes,arrows,shadows,mindmap,trees,calc}

%****************************************************************************************************

\setbeamercovered{dynamic}

\def\arrowd{
  (10.75:1.1) -- (6.5:1) arc (6.25:120:1) [rounded corners=0.5] --
  (120:0.9) [rounded corners=1] -- (130:1.1) [rounded corners=0.5] --
  (120:1.3) [sharp corners] -- (120:1.2) arc (120:5.25:1.2)
  [rounded corners=1] -- (10.75:1.1) -- (6.5:1) -- cycle
}

\tikzset{
  ashadow/.style={opacity=.25, shadow xshift=0.07, shadow yshift=-0.07},
}


%****************************************************************************************************
% COLORI e STILI di TESTO
\definecolor{greendark}{RGB}{0,178,140}
\definecolor{bluegreen}{cmyk}{0.9,0.0,0.35,0.2}
\definecolor{darkblue}{rgb}{0.2,0.2,0.65}
%\newcommand{\cit}{\scriptsize\color{bluegreen}}             % citazioni
\newcommand{\cit}{\scriptsize\color{themecolor!80!green}}             % citazioni
\newcommand{\tbtit}{\bf\color{themecolor!75!black}} % titoli nelle tabelle
\newcommand{\ev}{\color{themecolor}\bf}
\renewcommand{\CancelColor}{\color{red}}
%****************************************************************************************************


%****************************************************************************************************
% TABELLE
\newcolumntype{C}[1]{>{\centering\let\newline\\\arraybackslash\hspace{0pt}}m{#1}}
\setlength{\aboverulesep}{0pt}
\setlength{\belowrulesep}{0pt}
\setlength{\extrarowheight}{.75ex}
\arrayrulecolor{themecolor!75!black}

\newcommand{\tabitem}{~~\llap{\textbullet}~~}
%****************************************************************************************************

\pgfdeclarelayer{background}
\pgfdeclarelayer{foreground}
\pgfsetlayers{background,main,foreground}


%****************************************************************************************************
% ORBITALI
%****************************************************************************************************

% For black & white suggestions are black!95 for the electron
% and a white background, or a simple shade for the orbitals.
\colorlet{electron}{themecolor!75}
\tikzset{orbital/.style={thick,draw=themecolor,fill opacity=.60}}
% Styles for orbitals with 0, 1 and 2 atoms respectively.
\tikzset{orbital 0/.style={orbital,fill=themecolor!25}}
\tikzset{orbital 1/.style={orbital,fill=themecolor!66}}
\tikzset{orbital 2/.style={orbital,fill=themecolor}}
\tikzset{atomcore/.style={shape=circle,thick,draw=black!90,minimum size=20pt,
    font=\large\color{black!100!gray},fill=black!20,inner sep=0pt}}

\def\orbheight{1.3}%{1.2}
\def\orbwidth{0.5}%{.6}

% Parameters: #1 Rotation of the orbital
%             #2 Coordinate where the orbital should be attached
%             #3 Number of electrons to draw in the orbital
\newcommand{\orbital}[3]{
    \begin{scope}[rotate=#1,shift=(#2)]
        % These points define the curve for the orbital.
        \coordinate (c1) at (-\orbwidth, .6 * \orbheight);
        \coordinate (c2) at (-\orbwidth, \orbheight);
        \coordinate (c3) at (\orbwidth, \orbheight);
        \coordinate (c4) at (\orbwidth, .6 * \orbheight);
        \coordinate (top) at (0,\orbheight);

        %Coordinates of the electrons
        \coordinate (e1) at (0, 0.45*\orbheight);
        \coordinate (e2) at (0, 0.75*\orbheight);
    \end{scope}
  % These are drawn on a background layer, so orbitals
  % can overlap without covering the electrons, which
  % visualises the role electrons play in chemical bonds.
  \begin{pgfonlayer}{background}
      \draw[orbital #3] (#2) .. controls (c1) and (c2) .. (top) ..
            controls (c3) and (c4) .. (#2);
  \end{pgfonlayer}

  % Draw the electrons
  \ifnum#3>0
      \foreach \n in {1,...,#3} {
          \shade[ball color=electron] (e\n) circle (3pt);
      }
  \fi
}

% This allows to quickly place an atom.
% Parameters: #1 (Optional) Name of the center node
%             #2 Text for the center node
%             #3 A list of rotation-angle/anchor/number of electrons
\newcommand{\Atom}[3][AtomNode]{
  \node[atomcore] (#1) {\ce{#2}};
  \foreach \ang/\anchor/\n in {#3} {
      \orbital{\ang}{#1.\anchor}{\n}
  }
}
%****************************************************************************************************

%****************************************************************************************************
% FRECCE
\tikzset{bluearrow/.style={draw=themecolor,fill=themecolor,single arrow,drop shadow=%
                           {shadow xshift=.3ex,shadow yshift=-.3ex,color=themecolor!60!black},
                           minimum height=3.5ex,minimum width=0.1ex,single arrow head extend=0.5ex,
                           single arrow tip angle=70}}

\newcommand{\arrowup}{\tikz[baseline=-0.5ex]{\node[bluearrow,rotate=90]{};}}
\newcommand{\arrowdown}{\tikz[baseline=-0.5ex]{\node[bluearrow,rotate=-90]{};}}
\newcommand{\arrowright}{\tikz[baseline=-0.5ex]{\node[bluearrow]{};}}
\newcommand{\arrowleft}{\tikz[baseline=-0.5ex]{\node[bluearrow,rotate=180]{};}}
%****************************************************************************************************


% COMANDI PERSONALI
\newcommand{\gete}{\ce{GeTe}\xspace}

